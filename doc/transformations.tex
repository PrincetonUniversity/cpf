
\subsubsection{Memory Dependences}

Memory-related enabling transformations collect a set of memory objects
for which they are applicable (see ~\cref{enablers}).

\paragraph{Applicability:}
%
The applicability guard of a transformation uses the results of the
speculation-aware memory analyzer to determine which memory objects
satisfy the properties required by the transformation and records the
speculative assumptions used.
%
% XXX Make sure to add that assumptions need to be validated at runtime
% later on
% The applicability guard also records the speculative assumptions used
% to infer applicability for a memory object. These assumptions would
% need to be validated at runtime for the transformation to be correct.

\paragraph{Transformation proposal:} The output of each applicability
guard is assembled in a transformation proposal which is sent to the
DOALL planner (\cref{planner}).
%
The proposal includes for each memory object an estimated handling
cost based on the transformation itself and the validation cost of the
used speculative assumptions.
%
%The cost of each transformation is determined by the cost of the
%transformation itself and the cost of the speculative assumptions
%required for the transformation to be applicable.
%
For simplicity, each transformation and speculation
validation operation is assigned a fixed
%predefined
cost that ensures a basic ordering among the options. For
example,
%regarding speculative assumptions,
memory speculation has an extremely high cost (expensive validation),
loaded value prediction has a much smaller cost, while
control speculation has no cost.
%The estimated cost computation is very basic and just ensures basic
%ordering among the options.
%
%Every transformation is assigned an arbitrary cost that ensures some
%ordering among transformation. Speculative assumptions is similarly
%computed.
%
For the set of transformations and speculative assumptions in our
framework and in the context of DOALL parallelization, this simplified
cost model proved sufficient for detection of minimal cost plans.

%TODO: maybe add discussion about their cost
Enabling transformations address memory, register or/and control
cross-iteration dependences.

\begin{itemize}
%
\item Privatization: Discussed in ~\ref{novel_transf}

\item Reduction: Applicable for objects that only participate in
reduction operations. When applied, reallocates
its objects to the reduction heap. At commit, objects of parallel
workers are merged according to their reduction operation.

%Applicability guard: objects that only participate to reduction
%operations

%Transformation Application: only separate objects to reduction heap

%Runtime support: at commit merge all objects according to their
%reduction operation

\item Short-lived: Applicable for objects that only exist within one
iteration of the loop. Reallocates its objects to the
short-lived heap when applied and inserts check to ensure that all
the objects of the heap are freed at the end of each iteration.
% During runtime, the short-lived property is checked.

\item Read-only: Applicable for objects that are never written to within
the loop. Reallocates its objects to the read heap when applied. Requires no
transformation-specific checks.

\item IO deferral: Applicable for shared IO objects. When applied, it
replaces IO library calls with custom calls. During runtime, it
collects output operations and performs them in-order at commit.

\end{itemize}

\paragraph{Transformation Application:}
Each transformation reallocates
memory objects it is selected to handle
%If, during the planning phase, a transformation is selected for a set
%of memory objects, then during the code transformation phase, the
%transformation separates these objects
to its own heap, disjoint from any others; transformations may also
perform additional transformation-specific modifications.

Separating the objects is essential
for two reasons. First, each transformation may demand different
memory mapping semantics and handles objects differently at commit.
Second, mapping of memory accesses to
objects often relies on profiling information, especially in languages with
unrestricted pointers like C/C++. Ensuring that all objects' accesses
are contained within a transformation's heap is sufficient to
validate underlying object assumptions. This idea of separation has
been explored previously by Johnson et al. (Privateer~\cite{johnson:12:pldi}).
The speculative assumptions used for each selected transformation also need
to be validated at runtime to ensure correctness.
% The applicability guard also records the speculative assumptions used
% to infer applicability for a memory object. These assumptions would
% need to be validated at runtime for the transformation to be correct.
%Separation checks are required for memory accesses whose underlying
%objects are discovered via profiling.
%The runtime function for re-allocation of memory objects can be
%specialized by each transformation.
%
%Each transformation has its own runtime support for handling its
%objects during execution.

%The idea of separating memory objects has been explored previously by
%Johnson et al. (Privateer~\cite{johnson:12:pldi}).  However, Privateer
%employs a monolithic design that entangles classification of memory
%objects with memory speculation and other
%%monolithic design of memory object classification entangles
%%separation speculation with memory speculation and other
%profiling-based information, resulting in unnecessarily high runtime
%overheads (see example in section~\ref{motiv_example} and performance
%analysis in section ~\ref{eval}).
%%
%By contrast, \name employs a modular and extensible design that
%facilitates planning and selection of minimal cost solutions without
%unnecessary speculation.
%where enabling transformations offer to handle memory objects along
%with estimated costs -offers with costs and the selection happens.
%

%Every transformation can optimize the checks that are applied to its
%objects knowing that there is no aliasing with other objects handled
%by different transformations.

%ensures that all accesses of a memory object have the required
%properties.

% We describe in the next section efficient variants of the speculative
% privatization transformation, which constitute a contribution of this
% paper.  Other enabling transformations that are used in our framework
% but have been proposed in prior work are discussed in
% section \ref{design_transf}.
We propose the following enabling transformations for addressing memory,
register, and/or control cross-iteration dependences.

\subsubsection{Register \& Control Dependences}

Cross-iteration register dependences
%(not related to induction variables)
are handled with reduction, replication, control speculation or value
prediction. Replication replicates side-effect free computation across
parallel workers to overcome cross-iteration dependences and avoids
inter-thread communication.
%
Cross-iteration control dependences are handled either with
replication or control speculation.
%mention IV?
Use of replication allows handling of uncounted loops, namely loops
with unknown trip count when the loop is invoked.
%
In terms of transformation cost, all transformations have constant
cost except for replication.  Replication's cost depends on how many
instructions need to be replicated. Non-speculative enablers
(reduction and replication) are preferred, in most cases, over
speculative ones, and reduction is preferred over replication.

%Cross-iteration register and control dependences related to induction
%variables are handled differently if detected.

%handling live-outs, loop-carried
%
%In the case of speculative executive, registers with cross-iteration
%dependences and live-out registers
%are stored in memory at commit so that they are recoverable in case of
%misspeculation. For live-outs, storing to memory

%\begin{itemize}
%
%\item Scalar Reduction
%
%\item Replication: Replicate side-effect free computation across
%parallel workers.
%
%\end{itemize}
%
%\subsubsection{Control Dependences}
%
%\item Induction Variable Detection: Every worker gets assigned a
%
%\item Replication: Replicate cross-iteration control dependences for
%each worker
%
%\item Control Speculation: Speculate cross-iteration control
%dependences
