Transformation that modify the code to remove parallelization inhibitors.
There are memory, register and control dependences related ones.

Split to analysis and application of transformation parts.

\subsubsection{Memory Dependences}

Memory dependences Transfor

\paragraph{Classifier}
A classifier interprets analysis information, infers properties and classifies memory objects to
families.
%
Classifier is the analysis part of memory-related transformations.

show algorithm that classifies similar to Privateer

Each family and each used assumptions (validation) requires some transformations:
Privatization, reduction, control spec, separation speculation etc


mention stack allocation stuff.
After inlining, allocation happens outside the loop etc.

Mention peehole for private-checks

\subsubsection{Register Dependences}

Register dependences transformations:
Reduction,
Replicable

\subsubsection{Control Dependences}

control spec info incorporated into pdg. no classical transformation analysis
part here. Could remove this control deps section

\subsection{Loop Selection} An execution time profiler, similar to
gprof~\cite{Privateer26}, finds hot loops (at least 10\% of total
program execution).
%
For each hot loop, the compiler finds a DOALL parallelization plan and
estimates its profitability.
%
Out of the profitably parallelizable loops, certain loops are not
selected for parallelization. The excluded loops are either
simultaneously active with another more profitable loop (we do not
support nested parallelism) or their memory object assignments
conflict with the assignments of a more profitable loop (every memory
object can be allocated to only one family throughout the program in
our current implementation).


