\begin{abstract}

The promise of automatic parallelization, freeing programmers from the
error-prone and time-consuming process of making efficient use of
parallel processing resources, remains unrealized.  For decades,
memory analysis limited the applicability of non-speculative automatic
parallelization techniques (NS-APTs).  The introduction of speculative
automatic parallelization techniques (S-APTs) overcame these
applicability limitations using predictions in {\em independent}
speculative transformation passes.  This approach limits memory
analysis's understanding of code with speculation, leads to an
inefficient selection of S-APTs, and exhibits, even in the case of no
misspeculation, high communication and bookkeeping costs for validation
and commit.  This paper presents XXX, a speculative-DOALL
parallelization framework that exhibits the applicability of S-APTs
while approaching the efficiency of NS-APTs (when applicable).  XXX
works by exploring the impact of predictions in a pass integrating a
new speculation-aware memory analysis with APT cost-benefit estimation
to select an effective APT plan.  On 12 general-purpose C/C++
programs, compared to Privateer, a state-of-the-art speculative-DOALL
parallelization system, XXX demonstrates higher overall program
speedup (23.8$\times$ over sequential execution on a 28-core shared
memory machine vs.  11.9$\times$ by Privateer) by reducing speculative
parallelization overheads in ways not seen before.

%23.8$\times$ over sequential execution, 2$\times$ over Privateer, a
%state-of-the-art speculative-DOALL parallelization
%system.

\end{abstract}
