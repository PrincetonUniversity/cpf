
\subsubsection{Static Analysis}

\name uses a state-of-the-art memory dependence analysis framework
tailored for parallelization (\`{a} la CAF\cite{johnson:cgo:17}), in
which multiple simple analysis algorithms collaboratively attempt to
disprove dependences, and minimize the need for speculation.

Static analysis is also used to identify underlying memory objects of
memory operations and characterize dependences that cannot
be disproved (see section~\ref{overview_dependence_analysis}).
%In section~\ref{} we discussed how a dependence is characterized as
%\textit{overwrite}.

\paragraph{Extended Kill-Flow Analysis Algorithm:}
%\paragraph{Static Analysis Improvement}
Kill-Flow is a highly effective analysis algorithm that searches for
killing operations along all feasible paths between two operations. If
a killing operation is found, then these two operations cannot have a
dependence.  Since  there  may  be  infinitely  many  paths,  its
search is restricted to blocks which post-dominate the source of the
queried dependence and dominate the destination.
%
This approximation prevents detection of a common pattern (seen in
052.alvinn, 179.art, dijkstra) that can be observed in the code in
Figure~\ref{fig:dijkstra_motivation}.
%
Instruction in line 29 kills values flowing from the previous
iteration to the read in line 41.  However, there is no dominance
relation, and thus it is not be detected.
%
We extend Kill-Flow algorithm to detect this pattern.  Observe that
the loop header of the inner loop in line 28 dominates the read in
line 41.  The extended Kill-Flow treats this inner loop as a single
operation that overwrites a range of memory locations. This way, it
can easily be proven that this range write overwrites at every
iteration the memory addresses that are read in line 41.
%
%The effect of this extension on the running time of Kill-Flow is
%negligible.
%
This extension allows us to disprove additional data flows compared to
the state-of-the-art, and further reduce the need for memory
speculation.

\subsubsection{Speculation-aware Memory Analyzer}

Mention all the types of collab

mention unique paths
value-pred and
control spec


