\name uses a state-of-the-art memory dependence analysis framework
tailored for parallelization ~\cite{johnson:14:pldi}, in
which multiple simple analysis algorithms collaboratively attempt to
disprove dependences and minimize the need for speculation.
%
% Static analysis is also used to identify underlying memory objects of
% memory operations and characterize dependences that cannot
% be disproved (see section~\ref{overview_dependence_analysis}).
%In section~\ref{} we discussed how a dependence is characterized as
%\textit{overwrite}.

\paragraph{Extended Kill-Flow Analysis Algorithm:}
%\paragraph{Static Analysis Improvement}
Kill-Flow is a highly effective analysis algorithm that searches for
killing operations along all feasible paths between two operations. If
a killing operation is found, then these two operations cannot have a
dependence.  Since  there  may  be  infinitely  many  paths,  its
search is restricted to blocks which post-dominate the source of the
queried dependence and dominate the destination.
%
This approximation prevents detection of a common pattern (seen in
052.alvinn, 179.art, dijkstra) that can be observed in the code in
Figure~\ref{fig:dijkstra_motivation}.
%
The write to \texttt{\textbf{pathcost}} in line \textbf{29} kills
values flowing from the previous iteration to the read in line \textbf{41}.
However, there is no dominance
relation, and thus it cannot be detected.
%
We extend the Kill-Flow algorithm to detect this pattern. Observe that
the loop header of the inner loop in line \textbf{28} dominates the
read in line \textbf{41}. The extended Kill-Flow treats this inner loop as a single
operation that overwrites a range of memory locations. This way, it
can easily be proven that this range write overwrites
the memory addresses that are read in line \textbf{41} at every iteration.
%
%The effect of this extension on the running time of Kill-Flow is
%negligible.
%
This extension allows us to disprove additional data flows compared to
the state-of-the-art and further reduce the need for memory
speculation.
