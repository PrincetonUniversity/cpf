\section{Motivation}
\label{sec:motivation}

This work is motivated by the excessive use of memory speculation of
prior speculative systems that limited their efficiency.
%
In this section, we present a code example taken from MiBench~\cite{} benchmark
dijkstra (used in the evaluation of Privateer~\cite{}) to showcase how static
analysis along with cheap-to-validate speculative assumptions can infer
high-level program properties without the need for expensive memory speculation,
and enable scalable parallelization.

\lstset{basicstyle=\ttfamily, numbers=left, numberstyle=\tiny,
  stepnumber=1, numbersep=5pt}
\begin{figure*}[t]
  \centering
  \scriptsize
  \subfloat[dijkstra]
  {
    \label{fig:dijkstra}
    \begin{minipage}{7.5cm}
      \begin{lstlisting}[morekeywords={iPrev,piPrev,g_qCount},belowskip=0pt]
// global variables
int iPrev;
int g_qCount = 0;

void enqueue(..., int piPrev) {
  ...
  ... = piPrev;
  g_qCount++;
}

void dequeue(..., int *piPrev) {
  if (qHead) {
    ...
    *piPrev = ...
    g_qCount--;
  }
}

for (int i = 0; i < N; i++) { // hot loop
  ...
  enqueue(i, 0, NONE);
  while ( g_qCount ) {
    dequeue(&iNode, &iDist, &iPrev);
    for (k = 0; k < NUM_NODES; k++) {
      ...
      if ( valid node ) {
        ...
        enqueue(i, iDist+iCost, iNode);
      }
    }
  }
  ...
}

\end{lstlisting}

    \end{minipage}
  }
\end{figure*}

1) global variable g\_qCount

- speculative assumption:  g\_qCount is always zero at the beginning of the
iteration (value prediction)

- static analysis: alias analysis

- validation cost: perform load before loop exits or on backedges and compared
  predicted value with loaded value

-This value prediction can be seen as a store in the beginning of every
iteration that kills any data flow for this memory object from previous
iterations.  Any memory operation with a pointer must-aliasing with this
predicted global value address cannot source or sink RAW loop-carried memory
dependencies.  WAW loop-carried can also be ignored as the final content of this
memory location is predictable.  No need to perform any action to validate or
log (for last write) any of these individual memory operations.

2) global variable iPrev

- speculative assumption: the branch "if (qHead)" is always taken (control speculation)

- static analysis: alias analysis, killflow, noCaptureGlobal analysis passes

- validation cost: no validation cost for control spec (just misspecs if branch
  is not taken)

- Using the speculative assumption, killflow analysis can infer that the store
  in line 15 kills all other accesses of this global
variable (killed operations are identified by quering alias analysis). Given this
property, any RAW loop-carried dependence is disproved.  Additionally, static
analysis can enumerate all uses of this global, since it is not captured, and
detect that this global is used only within this loop, namely it is not a
live-out.  Thus, there is no need to log stores and keep track of the last
written value to it.

3) TODO: example with blackscholes

For all the above memory objects, Privateer~\cite{} would require expensive
logging on most of their accesses,
%either for validation checks or for identifying who wrote last what
yielding sub-optimal speedups as clearly
exhibited by our experimental results in section ~\ref{eval}.
%all these objects are classified as privatizable by Privateer

%showcases how fine-grained collaboration between static analysis and speculative
%assumptions can infer high-level program properties without the need for
%expensive memory speculation.

%use alvinn for value pred + alias analysis

%Note that WAR dependences can be ignored thanks to the process-based runtime
%system.


\centering
\tiny
\begin{tabular}{|c|c|c|c|c|c|c|c|c|}
  \hline
  Type of Private &
  Overlap         &
  Spec Privatization &
  Other Spec Allowed    &
  Loads Checks  &
  Store Checks    &
  Last-Write Detection &
  CoW Mapping &
  Copy-out to Main \\

%%%%%%%%%%%%%%%%%%% Shared %%%%%%%%%%%%%%%%%%%%%%%%%%
 \hline
 Shared & No & - & - & - & - & - & - & - \\
 \hline
%%%%%%%%%%%%%%%%%%% Local %%%%%%%%%%%%%%%%%%%%%%%%%%
 Local Private & Any & - & $\checkmark$ & - & - & - & $\checkmark$ & - \\
 \hline
%%%%%%%%%%%%%%%%%%% Overwrite %%%%%%%%%%%%%%%%%%%%%%%%%%
 Kill Private & Complete & - & $\checkmark$ & - & - & - & $\checkmark$ &
$\checkmark$ \\
 \hline
%%%%%%%%%%%%%%%%%%% Conservative Private %%%%%%%%%%%%%%%%%%%%%%%%%%
 Static Private & No/Partial & - & $\checkmark$ & - & - &
 $\checkmark$ & $\checkmark$ & $\checkmark$ \\
 \hline
%%%%%%%%%%%%%%%%%%% Specpriv Private %%%%%%%%%%%%%%%%%%%%%%%%%%
 Privateer Private & Any & $\checkmark$ & $\checkmark$ & $\checkmark$ & $\checkmark$ &
 $\checkmark$ & $\checkmark$ & $\checkmark$ \\
 \hline
\end{tabular}



%\lstset{basicstyle=\ttfamily, numbers=left, numberstyle=\tiny,
%  stepnumber=1, numbersep=5pt}
%\begin{figure*}[t]
%  \centering
%  \scriptsize
%  \subfloat[dijkstra -- predictable]
%  {
%    \label{fig:dijkstra}
%    \begin{minipage}{7.5cm}
%      \begin{lstlisting}[morekeywords={g_qCount},belowskip=0pt]
// predicted to be 0 at start of every iteration
int g_qCount = 0;

for ( int i = 0; int j = N/2; i < N; i++, j++ ) {
  ...
  enqueue( i, 0, NONE );
  g_qCount++;
  while ( g_qCount ) {
    dequeue( &iNode, &iDist, &iPrev );
    g_qCount--;
    for ( k = 0; k < NUM_NODES; k++ ) {
      ...
      if ( valid_node ) {
        ...
        enqueue( i, iDist + iCost, iNode );
        g_qCount++;
      }
    }
  }
  ...
}

\end{lstlisting}

%    \end{minipage}
%  }
%  \hspace{0.5cm}
%  \subfloat[dijkstra -- overwrite]
%  {
%    \label{fig:blackscholes}
%    \begin{minipage}{7.5cm}
%      \input{figures/dijkstra_overwrite_code}
%    \end{minipage}
%  }
%\end{figure*}
%
%
%
%
%\begin{figure*}[t]
%  \centering
%  \scriptsize
%  \subfloat[gemm -- shared]
%  {
%    \label{fig:gemm}
%    \begin{minipage}{7.5cm}
%      \begin{lstlisting}[escapeinside={~}{~}, belowskip=0pt]
for ( i = 0; i < ni; i++ ) {
  for ( j = 0; j < nj; j++ ) {
    C[i][j] *= beta; 
    for ( k = 0; k < nk; ++k )
      C[i][j] += alpha * A[i][k] * B[k][j];
  }
}
\end{lstlisting}

%    \end{minipage}
%  }
%  \hspace{0.5cm}
%  \subfloat[blackscholes -- overwrite]
%  {
%    \label{fig:blackscholes}
%    \begin{minipage}{7.5cm}
%      % look in Nick's motivation.tex for formatting
\begin{lstlisting}[morekeywords={prices}, belowskip=0pt]
for ( j = 0; j < NUM_RUNS; j++ ) {
  for ( i = 0; i < numOptions; i++ ) {
    prices[i] = BlkSchlsEqEuroNoDiv(
      sptprice[i], strike[i],
      rate[i], volatility[i], otime[i],
      otype[i], 0);
  }
}
\end{lstlisting}

%    \end{minipage}
%  }
%\end{figure*}
%\begin{figure*}[t]
%  \centering
%  \scriptsize
%  \subfloat[052.alvinn -- stack local]
%  {
%    \label{fig:alvinn_local}
%    \begin{minipage}{7.5cm}
%      \begin{lstlisting}[morekeywords={psum_array},belowskip=0pt]
float psum_array[NHU+1]; // stack

for (epoch = 0; epoch < NUM_EPOCHS; epoch++) {
  ...
  for ( i = 0; i < NHU+1; i++ )
    psum_array[i] = 0;
  for( i = 0; i < NOU; i++ ) {
    for ( j = 0; j < NHU+1; j++ ) {
      ...
      psum_array[j] += delta[i] * weights[i][j];
      ...
    }
  }
  for ( i = 0; i < NHU+1; i++ )
    delta[i] = hidden[i] * (1 - hidden[i]) * psum_array[i];
  ...
}
\end{lstlisting}

%    \end{minipage}
%  }
%  \hspace{0.5cm}
%  \subfloat[dijkstra -- global local]
%  {
%    \label{fig:dijkstra}
%    \begin{minipage}{7.5cm}
%      \begin{lstlisting}[morekeywords={k},belowskip=0pt]
int k;

for ( int i = 0; int j = N/2; i < N; i++, j++ ) {
  ...
  j = j%N;
  if ( i == j )
    continue;
  else {
    ...
    while ( g_qCount ) {
      ...
      for ( k = 0; k < NUM_NODES; k++ ) {
        ...
      }
    }
  }
  ...
}

\end{lstlisting}

%    \end{minipage}
%  }
%\end{figure*}

% \begin{figure*}[t]
%   \centering
%   \scriptsize
%   \subfloat[covariance -- nospec-private]
%   {
%     \label{fig:covariance}
%     \begin{minipage}{7.5cm}
%       \begin{lstlisting}[morekeywords={symmat}, belowskip=0pt]
for ( j1 = 0; j1 < m; j1++ ) {
  for ( j2 = j1; j2 < m; j2++ ) {
    symmat[j1][j2] = 0.0;
    for ( i = 0; i < n; i++ )
      symmat[j1][j2] += data[i][j1] * data[i][j2];
    symmat[j2][j1] = symmat[j1][j2];
  }
}
\end{lstlisting}

%     \end{minipage}
%   }
%   \hspace{0.5cm}
%   \subfloat[052.alvinn -- real private]
%   {
%     \label{fig:alvinn_specpriv}
%     \begin{minipage}{7.5cm}
%       \begin{lstlisting}[morekeywords={output_act}, belowskip=0pt]
float output_act[NOU];

for (epoch = 0; epoch < NUM_EPOCHS; epoch++) {
  ...
  receiver = &output_act[0];
  end_receiver = &output_act[NOU - 1];
  for ( ; receiver <= end_receiver; ) {
    *receiver = 0.0;
    sender = &hidden_act[0];
    end_sender= &hidden_act[NHU];
    for (; sender <= end_sender; )
      *receiver += (*sender++) * (*weight++);
    *receiver = SIGMOID(*receiver);
    *receiver++;
  }
  ...
  for (ou = 0; ou < NOU; ou++) {
  	delta[ou] = (teach[ou] - output_act[ou]) *
      output_act[ou] * (1.0 - output_act[ou]);
  	...
  }
  ...
}
\end{lstlisting}

%     \end{minipage}
%   }
% \end{figure*}
