% !TEX root = paper.tex
\section{Background and Motivation}
\label{sec:motivation}


Software-based automatic DOALL parallelization systems have been studied
for over two decades.
%
Early works including Polaris~\cite{blume:94:polaris},
SUIF~\cite{suif:94:stanford}, PD~\cite{rauchwerger:94:ics} and Hybrid
Analysis~\cite{rus:03:hybrid} use either static analysis to remove
dependences or run-time analysis to infer if the parallel code is safe to
execute. Privatization and reduction of these systems rely on
analysis to remove all flow dependences or prove the reduction property.
The drawback of static analysis is that it is too imprecise to parallelize
complex loops, while run-time analysis predicates can be too expensive to
extract. LRPD~\cite{dang:02:ipdps} and R-LRPD~\cite{dang:02:ipdps}
introduce speculative privatization, which executes the loop first and then
checks for violation of privatization criterion, to overcome static
analysis' limitations. However, all of these systems are not able to handle
dynamically allocated memory objects, which limits their applicability to
array-based scientific programs.

Recent works like STMlite~\cite{mehrara:09:stmlite}, Cluster
Spec-DOALL~\cite{kim:12:cgo}, Privateer~\cite{johnson:12:pldi} use
profile-guided speculation to remove specific dependences that static
analysis fails to address. This improves the applicability to loops with
pointers and dynamic data structures.  Among these works, Privateer
~\cite{johnson:12:pldi}, by using speculative heap separation, supports
both speculative privatization and reduction even in the presence of
unrestricted pointers, and enables greater applicability than other
automatic speculative-DOALL systems.

Despite that great applicability has been achieved, the results of
automatic speculative-DOALL systems on real hardware are still
underwhelming due to the existence of various overheads. There have been
attempts to reduce overheads. Cluster Spec-DOALL~\cite{kim:12:cgo} and
CorD~\cite{ctian:2008:micro} use static analysis to infer some properties
of an object to remove unnecessary memory speculation checks. Integrated in
Polaris~\cite{blume:94:polaris}, automatic array
privatization\cite{tu:94:lcpc} tries to avoid bookkeeping and merging
overheads specific to privatization. However, for most general purpose
programs, many workloads rely on irregular structures (e.g., trees, graphs,
lists) and static analysis are mostly not applicable to them. We believe
that there needs to be a sensible approach to automatic parallelization
that boosts efficiency without sacrificing applicability.


% \subsection{Difficulties of Reducing Overheads}


% Enabling transformations try to remove all dependences including flow (RAW)
% dependences and false (WAW and WAR) dependences. Without speculation, there
% are privatization and reduction transformations. With the introduction of
% speculation, there are choices like memory flow speculation, control
% speculation, and value prediction... What makes things more complex is
% all types of speculation can enable privatization and reduction. This is
% explored by Privateer as speculative privatization and speculative
% reduction.

% With so much information and options in hand, it's not trivial to decide
% how to resolve dependences. Prior techniques apply static analysis to generate
% a PDG, and use speculation to remove dependences, then as the last step,
% apply

% One insight is the applicability information of all types of speculations
% is available once the profilings are done, which precedes the main
% compilation flow. On contrary, privatization and reduction need help from
% analysis and speculation to decide whether they are applicable.


\subsection{Investigating Overheads}

\lstset{basicstyle=\ttfamily, numbers=left, numberstyle=\tiny,
  stepnumber=1, numbersep=5pt}
\begin{figure}[t]
  \centering
  %\begin{tabular}{cc}
  \scriptsize
%\resizebox{0.8\linewidth}{!}{
    \subfloat{
    \begin{minipage}{5cm}
      \begin{lstlisting}[morekeywords={pathcost,dist}, aboveskip=0pt, belowskip=0pt, firstnumber=1]
int *pathcost; // dyn alloc 1-D N
int *adj; // dyn alloc 2-D NxN
int dist;
int nDist;

void allocatePathCost() {
  pathcost = (int*)malloc(N*sizeof(int));
}

int dequeue() {
  if (!emptyQ()) {
\end{lstlisting}

\begin{lstlisting}[morekeywords={pathcost,dist}, aboveskip=0pt, belowskip=0pt,
firstnumber=13, name=dij_checks]
    dist = ...
    ...
  }
\end{lstlisting}

\begin{lstlisting}[morekeywords={pathcost,dist}, aboveskip=0pt, belowskip=0pt,
firstnumber=18, name=dij_checks]
}

void hot_loop(int N) {
\end{lstlisting}
\begin{lstlisting}[morekeywords={pathcost}, aboveskip=0pt, belowskip=0pt,
firstnumber=25]
  for (src=0; src<N; src++) {
\end{lstlisting}
\begin{lstlisting}[morekeywords={pathcost,dist}, aboveskip=0pt, belowskip=0pt,
firstnumber=28]
    for (i=0; i<N; i++)
      pathcost[i] = inf;

    enqueue(src, 0);
    while (!emptyQ()) {
      int v = dequeue();
      for (i=0; i<N; i++) {
\end{lstlisting}
\begin{lstlisting}[morekeywords={pathcost,dist}, aboveskip=0pt,
belowskip=0pt, firstnumber=38,name=dij_checks]
        nDist = adj[v][i] + dist;
\end{lstlisting}
\begin{lstlisting}[morekeywords={pathcost}, aboveskip=0pt, belowskip=0pt,
firstnumber=41]
        if (pathcost[i] > nDist) {
\end{lstlisting}
\begin{lstlisting}[morekeywords={pathcost}, aboveskip=0pt, belowskip=0pt,
firstnumber=44]
          pathcost[i] = nDist;
          enqueue(i, nDist);
        }
      }
    }
\end{lstlisting}
\begin{lstlisting}[morekeywords={pathcost}, aboveskip=0pt,
belowskip=0pt, firstnumber=52]
  }
\end{lstlisting}
\begin{lstlisting}[morekeywords={pathcost}, aboveskip=0pt,
belowskip=0pt, firstnumber=54]
}
\end{lstlisting}

    \end{minipage}
    }
\caption{Sequential \texttt{dijkstra} example from MiBench~\cite{guthaus:2001:iiwsc}}
\label{fig:dijkstra_motivation}
\end{figure}

Privateer is the state-of-the-art automatic parallelization system for the
workloads this paper targets. Hence, we investigated its results to find
the missing opportunities to reduce overheads.
%
Examining the evaluation of Privateer, we spotted two major overheads,
excessive use of memory speculation which is general for prior speculative
parallelization systems, and expensive privatization which applies to
most systems with privatization.
%
In \ref{sec:excessive-memory-spec} and \ref{sec:excessive-priv}, we present
the impact of this two main overheads on Privateer along with a motivating
example taken from the \texttt{dijkstra} benchmark (used in Privateer's
evaluation) from MiBench ~\cite{}. A simplified code of this example is
shown in Figure~\ref{fig:dijkstra_motivation}. The reuse of the array
\texttt{\textbf{pathcost}} and global variable \texttt{\textbf{dist}}
causes cross-iteration false dependence(s) that inhibit parallelization.
Privateer removes these dependences by creating private copies of
\texttt{\textbf{pathcost}} and \texttt{\textbf{dist}} for each worker.

\subsubsection{Excessive Use of Memory Speculation}
\label{sec:excessive-memory-spec}

Privateer's excessive use of memory speculation leads to large overheads
for monitoring reads of and writes to privatized objects, with an average
of 23.7 GB of reads and 18.4 GB of writes monitored per program reported in
the paper. \texttt{dijkstra} has particularly high overheads (84.9GB of
reads and 56.7GB of writes), which limits its speedup to 4.8$\times$ on 24
cores.


Privateer uses memory speculation to disprove the cross-iteration
RAW dependence from line \textbf{13} to line \textbf{37} in order to
privatize \texttt{\textbf{dist}}. Since the write to \texttt{\textbf{dist}}
is inside a conditional block, static analysis alone is unable to disprove
this dependence. Current speculative systems, including Privateer, will
first apply memory speculation to this dependence and then perform static
analysis, adding logging and validation (overheads). If, however, we can
use edge profiling to speculate that this conditional statement will always
execute and pass this speculative assumption to the static analysis, this
RAW dependence can be removed with minimal cost, avoiding all monitoring of
reads to this object. This combination is superior to using memory
speculation, as the cost for validating control speculation is negligible
compared to that of memory speculation.

\subsubsection{Expensive Privatization}
\label{sec:excessive-priv}
Perhaps counterintuitively, benchamrks that do not require memory
speculation still have heavy overhead due to bookkeeping of the write set
due to the inherent nature of privatization. For \texttt{dijkstra}, even
assuming all the private reads are removed, Privateer still need to log
56.8 GB of writes, which constitutes around 20\% of each parallel worker's
time. Apart from this overhead, privatization also introduces the cost of
merging the parallel workers' states during checkpointing. In
Figure~\ref{fig:dijkstra_motivation}, static analysis alone can disprove
all RAW dependences related to \texttt{\textbf{pathcost}} and safely
privatize it. However, because \texttt{\textbf{pathcost}} is a live-out
object, i.e., might be read after the loop invocation, each write to it
requires a log in order to determine which worker writes each byte of the
object. We can infer, though, that the \texttt{for} loop in lines
\textbf{28} and \textbf{29} will overwrite \texttt{\textbf{pathcost}} at
every iteration. Thus, the live-out values will only come from the last
iteration and no bookkeeping is needed.

\subsection{Motivation}
This work is motivated by the need of reducing significatly the two main
sources of overhead described above, namely the excessive use of memory
speculation and the expensive privatization, of prior speculative systems
that significantly limits their efficiency.
%
In the next section, we will show that by using speculation-aware memory
analyzer in conjuction with efficient privatization under a sensible
planner allows an efficient and scalable automatic parallelization while
maintaining the state-of-the-art applicability.



