The planner chooses the best performing set of transformations that
enables DOALL parallelization.
%
In DOALL parallelization each iteration need to run independently.
Thus, all the cross-iteration dependences need to be handled.
%
The input of the planner is the annotated loop PDG and the
transformation proposals.
%
The output is a selection of the best performing transformations that
address all the cross-iteration dependences. 
%with minimum estimated cost.


%The cross-iteration dependences of all memory objects need to be
%handled.  
%
The planner selects for each memory object the cheapest transformation
proposal that can address the cross-iteration dependences of this
object.
%If all the memory objects are handled by one single transformations,
%then there is no need for checking underlying objects.
%
%Each memory object can only be handled by one transformation.
%
For cross-iteration register and control dependences, the planner
examines individual dependences one by one and greedily selects the
cheapest cheapest option for each one.

The cost of each transformation is determined by the cost of the
transformation itself and the cost of the speculative assumptions
required for the transformation to be applicable. 
%
The estimated cost computation is very basic and just ensures basic
ordering among the options.
%
Every transformation is assigned an arbitrary cost that ensures some
ordering among transformation. Speculative assumptions is similarly
computed.
%
For the set of transformations and speculative assumptions in our
framework and in the context of DOALL parallelization simple ordering
proved sufficient.

%The selected set of transformations in the plan includes the selected
%transformations and validation code generator for their accompanying
%speculative assumptions.

Note that the planner produces non-speculative plans, if no speculative
assumptions are required for the DOALL parallelization.

%Planner decides the profitability, performing global reasoning instead
%of the traditional local reasoning of transformation when applied in
%a sequence.
