The planner chooses the best performing set of transformations that
enables DOALL parallelization.
%
The inputs of the planner are an annotated, using the
speculation-aware memory analyzer, loop PDG, and the transformation
proposals of the enabling transformations.
%
The output is a selection of the best performing transformations that
address all the cross-iteration dependences. 
%with minimum estimated cost.


In DOALL parallelization each iteration needs to run independently.
Thus, all the cross-iteration dependences need to be handled.
%
The planner selects for each memory object the cheapest transformation
proposal that can address the cross-iteration dependences of this
object.
%If all the memory objects are handled by one single transformations,
%then there is no need for checking underlying objects.
%
%Each memory object can only be handled by one transformation.
%
For cross-iteration register and control dependences, the planner
examines individual dependences one by one and greedily selects the
cheapest option for each one.  
%
If there is a memory object, or a cross-iteration register or control
dependences that cannot be addressed by any transformation then the
planner concludes that DOALL is not applicable.

%The selected set of transformations in the plan includes the selected
%transformations and validation code generator for their accompanying
%speculative assumptions.

The final plan is a set of enabling transformations and a set of
validation transformations for speculative assumption.  The plan also
specifies for each selected enabling transformation which memory
objects the transformation was selected for.
%
Note that the planner produces non-speculative plans, if no
speculative assumptions are required for DOALL parallelization.

%Planner decides the profitability, performing global reasoning instead
%of the traditional local reasoning of transformation when applied in
%a sequence.
