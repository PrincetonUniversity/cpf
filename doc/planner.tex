The planner chooses the best performing set of transformations that
enable DOALL parallelization.
%
The inputs of the planner are an annotated loop-centric PDG (generated
by the speculation-aware memory analyzer) and the transformation
proposals of the enabling transformations.
%
%
%In DOALL parallelization each iteration needs to run independently.
%Thus, all the cross-iteration dependences need to be handled.
%
The planner greedily selects the cheapest transformation proposal for
each memory object and for each cross-iteration register and control
dependence.
%for each memory object the cheapest transformation proposal that can
%address the cross-iteration dependences of this object.  %If all the
%memory objects are handled by one single transformations, %then there
%is no need for checking underlying objects.  % %Each memory object
%can only be handled by one transformation.  % For cross-iteration
%register and control dependences, the planner examines individual
%dependences one by one and greedily selects the cheapest option for
%each one.  
%
%
While simple, this approach is already enough to improve the
state-of-the-art of parallelizing compilers. 
%This is because \name has tightly merged speculation techniques with
%static analyses. 
Even though we can always increase the complexity of the planner, we
did not find yet empirical evidence that justify such extra
complexity. 
%
If there is a memory object, or a cross-iteration register or control
dependences that cannot be addressed by any enabling transformation
then the planner concludes that DOALL is not applicable.
%
%Planner decides the profitability, performing global reasoning
%instead of the traditional local reasoning of transformation when
%applied in a sequence.
%
The output of the planner is the best performing
%minimum overall estimated cost
set of transformations that address all the cross-iteration
dependences. 
%with minimum estimated cost.
%
%The selected set of transformations in the plan includes the selected
%transformations and validation code generator for their accompanying
%speculative assumptions.
%
The generated plan includes a set of enabling transformations and a
set of validation transformations for speculative assumptions.  The
plan also specifies for each selected enabling transformation which
memory objects or register/control cross-iteration dependences the
transformation was selected for.
%
Note that the planner produces non-speculative plans, if no
speculative assumptions are required for DOALL parallelization.
