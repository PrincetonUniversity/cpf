The runtime system of \name provides several benefits that allows for
low overheads for both speculative and non-speculative parallelization.

% Something about how killpriv only needs to copy out last checkpoint's
% values to main and no merging of checkpoints needed
\paragraph{Process spawning}
We use a process-based parallelism scheme as opposed to a thread-based one
for several purposes. The runtime system \texttt{fork()}s all the processes
only once at program startup to avoid the overhead of spawning at every loop
invocation, especially important for programs whose inner loops are
parallelized, such as \textit{052.alvinn}. Each worker process waits for the main
process to send only the function to run, the starting iteration, and the
size of each heap.

\paragraph{Shared memory}
At program startup, the main process creates and maps several regions of named shared
memory, with shared permissions, in \texttt{\textbf{/dev/shm}} corresponding
to each of the heap assignments.
When a worker starts an invocation, it remaps these regions of memory to
disjoint segments of its own virtual memory space with either
copy-on-write, shared, or read-only semantics depending on the heap
assignment. This in effect communicates all the live-in values from the
main process in an efficient manner. The separation of memory regions
allows for a very cheap heap check, only needing to perform a bitwise OR
with the most significant bits of the virtual address.
Each worker also allocates its own
region of shadow memory to log any accesses to the private heap at a
byte-level granuarity.

\paragraph{Checkpoints}
Each worker performs a privacy check on every private read or write to an
address with any of its accesses to the same address in a previous
iteration. It will also log in the shadow memory either the iteration of the
access in the case of a private write or a fixed value indicating a
private read occurred. Any violation of the privacy property will trigger a
misspeculation and a jump to recovery code. As \name enforces this
property at the byte-level, we perform a checkpoint every 252 iterations
before the shadow memory's iteration counter can overflow (255 possible
values per byte minus some values with fixed meaning \{XXX is this
clear?\}). The checkpoint reconciles all the worker's logs and checks for
misspeculation between iterations of different workers \{reword this???\} and
merges with previous checkpoints to save the current program state in the
case of misspeculation during the next checkpoint chunk. The overhead for
checkpointing is mainly dependent on the size of the reduction and private heaps,
as merging them requires performing the reduction or validation of privacy,
respectively.

As static analysis is used in conjunction with memory speculation, we can
infer that some benchmarks that were tested do not require any speculation.
We reallocate statically private objects (is this a thing?) to a new heap,
\texttt{KILL\_PRIVATE}, that does not require checks or logs on accesses or
merging during checkpoints. This allows us to reduce the overhead
drastically for non-speculative programs to achieve linear speedup.

% handling of live-ins, LC, live-outs

% lightweight,
% unified for both spec and non-spec


%The runtime system provides efficient mechanisms for replication of objects to
%support privatizationand for recovery from misspeculation
%
%By maintain-ing the same virtual address space with different mappings to
%phys-ical memory, the workers are able to operate on their privatemem-ory
%without  much  performance  penalty,  with  physical  pages  be-ing
%privatizedon-demandusing the Copy-On-Write mechanism
